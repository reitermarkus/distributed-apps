%% Begin slides template file
\documentclass[11pt,t,usepdftitle=false,aspectratio=169]{beamer}
\usepackage{graphicx}
\usepackage[utf8]{inputenc}
\usepackage{amsmath}
\usepackage{algorithm}
\usepackage[noend]{algpseudocode}
\usepackage{hyperref}
\usepackage[export]{adjustbox}

\makeatletter
\def\BState{\State\hskip-\ALG@thistlm}
\makeatother

%% ------------------------------------------------------------------
%% - aspectratio=43: Set paper aspect ratio to 4:3.
%% - aspectratio=169: Set paper aspect ratio to 16:9.
%% ------------------------------------------------------------------

\usetheme[nototalframenumber,foot,logo]{uibk}
\setbeamertemplate{bibliography item}[text]

%% ------------------------------------------------------------------
%% - foot: Add a footer line for conference name and date.
%% - logo: Add the university logo in the footer (only if 'foot' set).
%% - bigfoot/sasquatch: Larger font size in footer.
%% - nototalslidenumber: Hide the total number of slides (only if 'foot' set)
%% - license: Add CC-BY license symbol to title slide (e.g., for conference uploads)
%% - licenseall: Add CC-BY license symbol to all subsequent slides slides
%% - url: use \url{} rather than \href{} on the title page
%% ------------------------------------------------------------------

%% ------------------------------------------------------------------
%% The official corporate colors of the university are predefined and
%% can be used for e.g., highlighting something. Simply use
%% \color{uibkorange} or \begin{color}{uibkorange} ... \end{color}
%% Defined colors are:
%% - uibkblue, uibkbluel, uibkorange, uibkorangel, uibkgray, uibkgraym, uibkgrayl
%% The frametitle color can be easily adjusted e.g., to black with
%% \setbeamercolor{titlelike}{fg=black}
%% ------------------------------------------------------------------

%\setbeamercolor{verbcolor}{fg=uibkorange}
%% ------------------------------------------------------------------
%% Setting a highlight color for verbatim output such as from
%% the commands \pkg, \email, \file, \dataset
%% ------------------------------------------------------------------


%% information for the title page ('short title' is the pdf-title that is shown in viewer's titlebar)
\title[Distributed Applications - Project Presentation]{Distributed Applications - Project Presentation}
\subtitle{Project N  - Title, 28.01.2020}
%%\URL{www.uibk.ac.at/statistics}

\author[Author 1 \& Author 2]{Author 1 \& Author 2}
%('short author' is the pdf-metadata Author)
%% If multiple authors are required and the font size is too large you
%% can overrule the font size of author and url by calling:
%\setbeamerfont{author}{size*={10pt}{10pt},series=\mdseries}
%\setbeamerfont{url}{size*={10pt}{10pt},series=\mdseries}
%\URL{}
%\subtitle{}

\footertext{Distributed Applications - Project Presentation Template - Team M}
%\date{2019-05-07}

\headerimage{4}
%% ------------------------------------------------------------------
%% The theme offers four different header images based on the
%% corporate design of the university of innsbruck. Currently
%% 1, 2, 3 and 4 is allowed as input to \headerimage{...}. Default
%% or fallback is '1'.
%% ------------------------------------------------------------------

\begin{document}

%% ALTERNATIVE TITLEPAGE
%% The next block is how you add a titlepage with the 'nosectiontitlepage' option, which switches off
%% the default behavior of creating a titlepage every time a \section{} is defined.
%% Then you can use \section{} as it's originally intended, including a table of contents.
% \usebackgroundtemplate{\includegraphics[width=\paperwidth,height=\paperheight]{titlebackground.pdf}}
% \begin{frame}[plain]
%     \titlepage
% \end{frame}
% \addtocounter{framenumber}{-1}
% \usebackgroundtemplate{}}

%% Table of Contents, if wanted:
%% this requires the 'nosectiontitlepage' option and setting \section{}'s as you want them to appear here.
%% Subsections and subordinates are suppressed in the .sty at the moment, search
%% for \setbeamertemplate{subsection} and replace the empty {} with whatever you want.
%% Although it's probably too much for a presentation, maybe for a lecture.
% \begin{frame}
%     \vspace*{1cm plus 1fil}
%     \tableofcontents
%     \vspace*{0cm plus 1fil}
% \end{frame}


%% this sets the first PDF bookmark and triggers generation of the title page
\section{Bookmark Title}

\begin{frame}
\frametitle{Project Report}
    Team Members
    \begin{itemize}
        \item Author 1
        \item Author 2
    \end{itemize}
\end{frame}

\begin{frame}
\frametitle{Introduction}
    \begin{itemize}
        \item Project number
        \item Programming language
        \item Problem motivation
    \end{itemize}
\end{frame}

\begin{frame}
\frametitle{Wokflow Structure (AFCL)}
    \begin{itemize}
        \item Screenshot from FC Editor
        \item Data inputs
        \item Data outputs
        \item Functions
    \end{itemize}
\end{frame}

\begin{frame}
\frametitle{Scheduler to CFCL}
    \begin{itemize}
        \item Scheduler algorithm and implementation
        \item Screenshot from FC Editor
    \end{itemize}
\end{frame}

\begin{frame}
\frametitle{Helper (automation) tools (developed by you)}
    \begin{itemize}
        \item If you developed additional tools or used existing to automate, show them on this slide
        \item Mark maybe with different color in which parts is the automation
    \end{itemize}
    \vspace{1cm}
    // check graphic from PPT template
\end{frame}

\begin{frame}
\frametitle{Main Challenges}
    \begin{itemize}
        \item Emphasize the main problem(s) you faced
        \item How did you handle the problem(s)
        \item Other relevant challenges
    \end{itemize}
\end{frame}

\begin{frame}
\frametitle{Demo}
    \begin{itemize}
        \item Present that your FC works
    \end{itemize}
\end{frame}

\begin{frame}
\frametitle{Testing Methodology}
    \begin{itemize}
        \item Experiment setup
        \item If you made some changes, justify them here
    \end{itemize}
\end{frame}

\begin{frame}
\frametitle{Evaluation}
    \begin{itemize}
        \item Present the speedup of distributed vs. sequential
        \item Explain whether the speedup / slowdown is expected / unexpected
    \end{itemize}
\end{frame}

\begin{frame}
\frametitle{Recommendation}
\end{frame}


%% to show a last slide similar to the title slide: information for the last page
%%\title{Thank you for your attention!}
%%\subtitle{}
%%\section{Thanks}


%% appendix of 'extra' slides
%%\appendix
\begingroup

{\footnotesize
\begin{frame}
	\frametitle{References}
	\begin{minipage}[t]{1\textwidth}

      \begin{thebibliography}{9}

        \bibliofont

        \bibitem{afcl}
		Ristov, S., Pedratscher, S. and Fahringer, T.
		\textbf{AFCL: An Abstract Function Choreography Language for serverless workflow specification.}
		\textit{Future Generation Computer Systems vol. 114, p. 368--382}, 2020.

	  \end{thebibliography}
      \vspace{1cm}
   \end{minipage}
\end{frame}
}

%% Additional slides
\begin{frame}
    \frametitle{Appendix}

\end{frame}
\endgroup

\end{document}
