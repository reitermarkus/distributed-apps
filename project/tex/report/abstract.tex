\noindent

Serverless Computing has seen immense growth in the past couple of years and saw
many useful applications. Its intuitive design philosophy opens many
opportunities for developers to build and deploy scalable applications that can
run on any device, which is especially important in the current climate where
mobile and IoT devices dominate the market. While serverless computing has it's
benefits it certainly also has it's drawbacks depending on the use case.
Fortunately, we will see both, which makes for a better evaluation of the
paradigm.

In this report we test the use of serverless computing for stock price
prediction. For this we have created a FC with 4 functions, to first fetch the
prices, followed by a functions who predicts the stock price according to the
fetched prices, proceeded then by function, which processes the results and
finally a function, that creates a chart with \texttt{quickchart.io}. The functions
itself are written in two languages each, Rust and NodeJS with TypeScript for
the type checking. The reason for choosing the former is because of it's unique
ownership models, that allows for writing performant and save code, which is
especially important with serverless computing. Choosing the latter of the two
programming language is because of it being the prime language for FaaS
programming with its great integration in basically every cloud.

Once the functions are ready, they are deployed on the IBM cloud. While the
functions are deployed on IBM, there is still a dependency on AWS, since the
function for predicting the stock price relies on it. This function is
especially challenging to handle for serverless functions due to the forecast's
long running time. The forecast function's main drawback shines a light on the
main benefit of the FC, since the forecast runs in parallel for each stock,
which mitigates the losses in time at some point.

\textcolor{magenta}{Tentative abstract length: between a half and one page.}

% remove vspace in your final version
\vspace{20 pt}

\\The abstract should provide a brief summary of the report. Please provide details on following topics:

\\\\\textbf{Project}:
Introduce and motivate the use case you addressed in the project.
Please provide (a)~a brief description of the FC you were working with, (b)~information about the programming language use used to implement the functions, and (c)~a brief motivation for the deployment of the FC across multiple FaaS systems (regions).

% remove vspace in your final version
\vspace{20 pt}

\\\textbf{Methods}: Briefly describe which cloud services, FaaS systems, regions, and function deployments have you used for the implementation.

% remove vspace in your final version
\vspace{20 pt}

\\\textbf{Challenges}: Briefly describe the main challenge(s) that you faced during the project in terms of development, deployment, or execution (e.g., FaaS system limitations, AFCL limitations, etc).


% remove vspace in your final version
\vspace{10 pt}

\\\textbf{Main results}: Discuss the benefit / drawback of distributing the work as an FC (e.g., achieved speedup / slowdown) across multiple FaaS systems regions.

