\chapter{\label{chap:solution}Solution}


%
%
%
\section{FC Implementation in AFCL}


\begin{figure}[h]
  \centering
  \includegraphics[height=12cm, keepaspectratio]{./assets/afcl}
  \caption{AFCL}
  \label{fig:afcl}
\end{figure}

As can be seen in \cref{fig:afcl}, our function choreography consists of
four steps.

As input, the FC receives two variables: A \texttt{symbols} array containing
the ticker symbols and a \texttt{symbol\_count} number.

The first two steps can be done in parallel: The \texttt{fetchProcess} function
receives a single \texttt{symbol} and fetches the prices for this symbol by
querying the Alphavantage API and stores the result in an IBM bucket. It then
passes the \texttt{symbol} to the output and also adds a \texttt{object\_key}
to the output, which indicates the file name in the bucket.

The \texttt{forecast} function then receives that output as input and creates a
forecast using AWS Forecast: The prices are first converted from JSON into CSV format
and stored on an AWS bucket which can be accessed by AWS Forecast. It then creates
a dataset group, a dataset, a data import job, a predictor and finally a forecast.
The last step is to create a query to retrieve the forecast data. The result is
stored in an IBM bucket. The output of this function again contains the \texttt{symbol}
as well as the forecast data as an array, in our case only containing a single
element for the current day.

Once the \texttt{parallelFor} is finished, all \texttt{forecast} outputs are passed
to the \texttt{processResult} function, which first retrieves the given \texttt{object\_keys}
from the IBM bucket and then transforms the data into the following format for
\texttt{quickchart.io}: A \texttt{labels} array, containing the data dates and a
\texttt{datasets} array, which contains elements with a \texttt{symbol} and a
\texttt{data} array with the forecast prices.

Finally, the \texttt{create\_chart} function receives this output, makes an API
call to \texttt{quickchart.io} and eventually returns a \texttt{url} as output.

Overall, the complexity of our FC is O(n), since the FC loops over all ticker symbols
given as input. This means that the FC can be easily distributed by simply distributing
the ticker symbols among different function deployments.



%
%
%
\section{Scheduler}

Our scheduler receives a AFCL YAML file as its input and in addition, the number of iterations and the concurrency limit.
As output, we directly create a YAML file as well.

For the implementation of our scheduler, we chose Rust. We created structs representing all the parts of the YAML and
used the \texttt{serde} library to serialise and deserialise them.

This way we can parse a full AFCL YAML file as input. Our program them loops through the \texttt{workflowBody} and
for each \texttt{parallelFor}, it applies the following algorithm:

\begin{itemize}
  \item Inputs: \texttt{iterations} and \texttt{concurrency\_limit}
  \item Variables: \texttt{concurrency\_limits} (\texttt{Map<String, usize>}), \\ \texttt{function\_iterations} (\texttt{Map<String, usize>}) and \\ \texttt{est} (\texttt{Map<String, (f64, f64)>})
  \item For \texttt{i} in \texttt{0...iterations}, loop through functions in \texttt{parallelFor} block:
    \begin{enumerate}
      \item Look up FDs in database and insert function names in all maps if not already contained
      \item Sort FDs by the start time stored in \texttt{est} and loop through them:
        \begin{enumerate}
          \item If FD has not yet reached the \texttt{concurrency\_limit} in \texttt{concurrency\_limits}, select it and update \texttt{concurrency\_limits}.
          \item Otherwise, reset the limit and increase the start time in \texttt{est}.
        \end{enumerate}
  \end{enumerate}
\end{itemize}

After running the algorithm, we know how to split a \texttt{parallelFor} into multiple \texttt{parallelFor}s and
create a new struct containing a \texttt{parallel} block with nested \texttt{parallelFor} blocks. The result is a
CFCL using the same structure, which means we can easily output it directly in YAML format.



%
%
%
\section{FC Implementation in CFCL}

\begin{figure}[h]
  \centering
  \includegraphics[height=12cm, keepaspectratio]{./assets/cfcl}
  \caption{CFCL}
  \label{fig:cfcl}
\end{figure}

As can be seen in \cref{fig:cfcl}, the single \texttt{parallelFor} from \cref{fig:afcl} has
been changed to a \texttt{parallel} section with two \texttt{parallelFor}s.

With 20 iterations and a concurrency limit of 2, 12 iterations of the \texttt{fetch\_prices\_js} and \texttt{forecast\_js}
functions were chosen for the first \texttt{parallelFor}. For the second \texttt{parallelFor},
8 iterations of the \texttt{fetch\_prices\_rs} and \texttt{forecast\_rs} functions were chosen.

% TODO: Discuss the CFCL complexity and expected theoretical speedup.



%
%
%
\section{Automatic Deployment}

For deployment, we used Terraform. We use it to deploy 8 functions to IBM cloud and
to create a bucket for use with AWS Forecast. Additionally, we created two Bash scripts
for compiling and packaging our Rust and TypeScript functions. We also use it to create
a \texttt{.env} which contains secrets and is used when compiling the functions.



%
%
%
\section{Helper tools from you}

Mention developed tools by you to help you to manage:
\begin{itemize}
    \item logs from \textit{xAFCL}
    \item parse AFCL and create CFCL
    \item parse AFCL and create CFCL
\end{itemize}
