%
%
%
\chapter{\label{chap:introduction}Introduction}

Serverless Computing is a versatile paradigm with many use cases, which gained
in popularity in the recent past.  Especially since the mobile revolution more
and more opportunities open themselves up for the usage of FaaS applications.  In
this project we investigated the feasability of using Serverless Computing for
stock prediction. The stock marketing is interesting as it is to some extend
unpredictable and therefore a little help for investing is welcome. For this
reason the core of the project is to use AWS Forecast to predict future prices
of stocks in order to give investors a better understanding of the possible
outcome.

AWS Forecast uses machine learning to predict the prices therefore data is
needed for it to function properly. To do so first some previous stock prices
are required. Those can be acquired with the AlphaVantage API.
AlphaVantage makes it easy to retrieve the daily time series over period of
several months. This should provide AWS Forecast with enough data for an
adequate prediction. After the prediction the results will be processed in
order for them to be presented in a chart. More specifically for the chart a
convenient service with the name \texttt{quickchart.io} will be used that
makes that process easy and straight forward. It takes a JSON in the
body of the request and creates the chart accordingly. The following functions
are the core components of the FC:

\begin{itemize}
  \item \texttt{fetchPrices}
  \item \texttt{forecast}
  \item \texttt{processResults}
  \item \texttt{createChart}
\end{itemize}

In our AFCL \texttt{fetchPrices} followed by \texttt{forecast} will be executed
in parallel over all the processed stocks. The result of those will then be
used by \texttt{processResults} to create a chart. This increases the executed
time of the whole choreography considerably because the forecast function takes
quite some time.

All functions are deployed on the IBM cloud and are
implemented in the Rust programming language as well as in
NodeJS with TypeScript as the type checker. This gives an
interesting opportunity to compare both languages regarding their suitability
for the task at hand. It also offers an insight on how each language handles
serialisation as well as interfering with all the required interfaces in order
to get a final result.
